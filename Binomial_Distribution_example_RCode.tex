

\title[MA4413]{Statistics for Computing \\ {\normalsize MA4413 Lecture 4A}}
\author[Kevin O'Brien]{Kevin O'Brien \\ {\scriptsize Kevin.obrien@ul.ie}}
\date{Autumn Semester 2011}
\institute[Maths \& Stats]{Dept. of Mathematics \& Statistics, \\ University \textit{of} Limerick}

\renewcommand{\arraystretch}{1.5}

\begin{document}


\titlepage



\frametitle{The Binomial Probability Distribution}
<p>
* The number of independent trials is denoted $n$.
* The probability of a `success' is $p$
* The expected number of `successes' from $n$ trials is $E(X) = np$
\end{itemize}
}
%---------------------------------------------------------------------------%

\frametitle{Characteristics of a Poisson Experiment}
A Poisson experiment is a statistical experiment that has the following properties:
<p>
* The experiment results in outcomes that can be classified as successes or failures.
* The average number of successes (m) that occurs in a specified region is known.
* The probability that a success will occur is proportional to the size of the region.
* The probability that a success will occur in an extremely small region is virtually zero.
\end{itemize}
Note that the specified region could take many forms. For instance, it could be a length, an area, a volume, a period of time, etc.
}

%---------------------------------------------------------------------------%

\frametitle{Poisson Distribution}


}

%---------------------------------------------------------------------------%

\frametitle{The Poisson Probability Distribution}
<p>
* A Poisson random variable is the number of successes that result from a Poisson experiment.
* The probability distribution of a Poisson random variable is called a Poisson distribution.
* This distribution describes the number of occurrences in a unit period (or space)
* The expected number of occurrences is $m$
\end{itemize}
}

%---------------------------------------------------------------------------%

\frametitle{Poisson Formulae}
The probability that there will be $k$ occurences in a unit time period is denoted $P(X=k)$, and is computed as follows.
\Large
\[ P(X = k)=\frac{m^k e^{-m}}{k!} \]

}
%---------------------------------------------------------------------------%

\frametitle{Poisson Formulae}
Given that there is on average 2 occurrences per hour, what is the probability of no occurrences in the next hour? \\ i.e. Compute $P(X=0)$ given that $m=2$
\Large
\[ P(X = 0)=\frac{2^0 e^{-2}}{0!} \]
\normalsize
<p>
* $2^0$ = 1
* $0!$ = 1
\end{itemize}
The equation reduces to
\[ P(X = 0)=e^{-2} = 0.1353\]
}
%---------------------------------------------------------------------------%

\frametitle{Poisson Formulae}
What is the probability of one occurences in the next hour? \\ i.e. Compute $P(X=1)$ given that $m=2$
\Large
\[ P(X = 1)=\frac{2^1 e^{-2}}{1!} \]
\normalsize
<p>
* $2^1$ = 2
* $1!$ = 1
\end{itemize}
The equation reduces to
\[ P(X = 1) = 2 \times e^{-2} = 0.2706\]
}
%---------------------------------------------------------------------------%

\frametitle{Continuous Random Variables}

<p>
* Probability Density Function
* Cumulative Density Function
\end{itemize}


If X is a continuous random variable then we can say that the probability of obtaining a \textbf{precise} value $x$ is infinitely small, i.e. close to zero.

\[P(X=x) \approx 0 \]

Consequently, for continuous random variables (only),  $P(X \leq x)$ and $P(X < x)$ can be used interchangeably.

\[P(X \leq x) \approx P(X < x) \]


}

%---------------------------------------------------------------------------%

\frametitle{Continuous Uniform Distribution}
A random variable X is called a continuous uniform random variable over the interval $(a,b)$ if it's probability density function is given by

\[ f_{X}(x)  =  { 1 \over b-a}   \hspace{2cm}  \mbox{ when } a \leq x \leq b\]

The corresponding cumulative density function is

\[ F_x(x) = { x-a \over b-a}   \hspace{2cm}  \mbox{ when } a \leq x \leq b\]

}

%-----------------------------------------------------------------------------%



The mean of the continuous uniform distribution is

\[ E(X) = {a+b \over 2}\]

\[ V(X) = {(b-a)^2\over12}\]
}

%-----------------------------------------------------------------------------%


\frametitle{The Memoryless property}
The most interesting property of the exponential distribution is the \textbf{\emph{memoryless}} property. By this , we mean that if  the lifetime of a component is exponentially distributed, then an item which has been in use for some time is a good as a brand new item with regards to the likelihood of failure.

The exponential distribution is the only distribution that has this property.
}

%--------------------------------------------------------%


\frametitle{Random Variables}
A pair of dice is thrown. Let X denote the minimum of the two numbers which occur.
Find the distributions and expected value of X.
}
%-------------------------------------------------------------%

\frametitle{Random Variables}
A fair coin is tossed four times.
Let X denote the longest string of heads.
Find the distribution and expectation of X.}
%-------------------------------------------------------------%
\frametitle{Random Variables}
A fair coin is tossed until a head or five tails occurs.
Find the expected number E of tosses of the coin.}
%-------------------------------------------------------------%
\frametitle{Random Variables}A coin is weighted so that P(H) = 0.75 and P(T ) = 0.25

The coin is tossed three times. Let X denote the number of
heads that appear.
<p>
* (a) Find the distribution f of X.
* (b) Find the expectation E(X).
\end{itemize}
}

%-------------------------------------------------------------%

<p>
* Now consider an experiment with only two outcomes. Independent repeated trials of such an experiment are
called Bernoulli trials, named after the Swiss mathematician Jacob Bernoulli (1654�1705). * The term \textbf{\emph{independent
trials}} means that the outcome of any trial does not depend on the previous outcomes (such as tossing a coin).
* We will call one of the outcomes the ``success" and the other outcome the ``failure".
\end{itemize}
}

%-------------------------------------------------------------%

<p>
 \item
Let $p$ denote the probability of success in a Bernoulli trial, and so $q = 1 - p$ is the probability of failure.
A binomial experiment consists of a fixed number of Bernoulli trials. * A binomial experiment with $n$ trials and
probability $p$ of success will be denoted by
\[B(n, p)\]
\end{itemize}
}
%-------------------------------------------------------------%

%---------------------------------------------------------------------------%

\frametitle{Probability Mass Function}
<p> * a probability mass function (pmf) is a function that gives the probability that a discrete random variable is exactly equal to some value. * The probability mass function is often the primary means of defining a discrete probability distribution \end{itemize}
}
%------------------------------------------------------------------%

Thirty-eight students took the test. The X-axis shows various intervals of scores (the interval labeled 35 includes any score from 32.5 to 37.5). The Y-axis shows the number of students scoring in the interval or below the interval.

\textbf{\emph{cumulative frequency distribution}}A  can show either the actual frequencies at or below each interval (as shown here) or the percentage of the scores at or below each interval. The plot can be a histogram as shown here or a polygon.
}




\end{document}



%---------------------------------------------------------------------------------------------------------------%
%----R Code ----
%---------------------------------------------------------------------------------------------------------------%
n=60000
Y=numeric(n)
for ( i in 1:n){

X=floor(runif(100,1,7))
Y[i]=sum(X)
}

Y
hist(Y,breaks=seq(300,400,by=10),main=c("Totals of 100 Die Throws"),cex.lab=1.4,font.lab=2,xlab=c("Total Score"))

hist(Y,breaks=seq(300,400,by=20),main=c("Totals of 100 Die Throws"),cex.lab=1.4,font.lab=2,xlab=c("Total Score"))



Z=seq(1:n)
Y/Z

plot(Y/Z,type="l",col="red",main=c("Die Rolls: Running Average"),font.lab=2,ylab="Average Value", xlab=
" Number of Throws")
abline(h=3.5,col="green")


#####################################################

plot(Z,Z.y,pch=16,col="red",ylim=c(2.5,5.5),main=c("Variance"),font.lab=2,ylab=" ", xlab="X: Green  Y: Blue  Z: Red" )

points(Y,Y.y,pch=16,col="blue" )
points(X,X.y,pch=16,col="green" )
points(c(1000,1000,1000),c(3,4,5),pch=18,cex=1.2)
lines(c(1000,1000),c(2.75,5.25),lty=3)



n=100000
Y=numeric(n)
for ( i in 1:n){

X=floor(runif(100,1,7))
Y[i]=sum(X)
}

Y
hist(Y,breaks=seq(270,430,by=2),main=c("Totals of 100 Die Throws (n= 100,000)"),cex.lab=1.4,font.lab=2,xlab=c("Total Score")) 



\titlepage



\frametitle{The Binomial Probability Distribution}
A Quick Review of the Binomial Distribution
<p>
* The number of independent trials is denoted $n$.
* The outcome of interest is known as a ``Success".
* The other outcome is known as a ``failure".  
* Often the applications of these names is counter-intuitive, i.e. defective components being the ``success".
* The probability of a `success' is $p$ 
* The expected number of `successes' from $n$ trials is $E(X) = np$
* The \texttt{binom} family of commands in \texttt{R} are what we use to compute necessary values.
\end{itemize}
}

\end{document}



%---------------------------------------------------------------------------------------------------------------%
%----R Code ----
%---------------------------------------------------------------------------------------------------------------%
n=60000
Y=numeric(n)
for ( i in 1:n){

X=floor(runif(100,1,7))
Y[i]=sum(X)
}

Y
hist(Y,breaks=seq(300,400,by=10),main=c("Totals of 100 Die Throws"),cex.lab=1.4,font.lab=2,xlab=c("Total Score"))

hist(Y,breaks=seq(300,400,by=20),main=c("Totals of 100 Die Throws"),cex.lab=1.4,font.lab=2,xlab=c("Total Score"))



Z=seq(1:n)
Y/Z

plot(Y/Z,type="l",col="red",main=c("Die Rolls: Running Average"),font.lab=2,ylab="Average Value", xlab=
" Number of Throws")
abline(h=3.5,col="green")


#####################################################

plot(Z,Z.y,pch=16,col="red",ylim=c(2.5,5.5),main=c("Variance"),font.lab=2,ylab=" ", xlab="X: Green  Y: Blue  Z: Red" )

points(Y,Y.y,pch=16,col="blue" )
points(X,X.y,pch=16,col="green" )
points(c(1000,1000,1000),c(3,4,5),pch=18,cex=1.2)
lines(c(1000,1000),c(2.75,5.25),lty=3)



n=100000
Y=numeric(n)
for ( i in 1:n){

X=floor(runif(100,1,7))
Y[i]=sum(X)
}

Y
hist(Y,breaks=seq(270,430,by=2),main=c("Totals of 100 Die Throws (n= 100,000)"),cex.lab=1.4,font.lab=2,xlab=c("Total Score")) 


<p>
* Binomial Coefficients / The Choose Operator
* Definition: The Probability Mass Functions (pmf)
* Binomial Distribution : Example
\end{itemize}
}

\frametitle{Binomial Coefficients}

In the last class, we came across binomial coefficients. Informally, binomial coefficients are the number of ways $k$ items can be selected from a group of $n$ items. 
The binomial coefficient indexed by n and k is usually written as $^nC_k$ or
\[ {n \choose k}\].
$C$ is colloqially known as the ``choose operator".

\[ {n \choose k} = \frac{n!}{k! \times (n-k)!} \]

(We call the operator the choose operator. We will use both notations interchangeably.)
 

}

\frametitle{Binomial Coefficients}

<p>
* $n!$ and $k!$ are the coefficients of $n$ and $k$ respectively.
* $n! = n \times (n-1) \times (n-2) \times \ldots \times 2 \times 1$
* For example $5! = 5\times4\times3\times2\times1 = 120$
* $n! = n \times (n-1)!$
* Importantly $0! = 1$ not 0.
\end{itemize}
\[ {6 \choose 2} = \frac{6!}{2! \times (6-2)!} = \frac{6!}{2! \times 4!}  \]\[\mbox{   } = \frac{6 \times 5 \times 4!}{2! \times 4!} 
 = 30/2 =15 \]
More examples of Binomial coefficients on blackboard.
}
%---------------------------------------------------------------------------%

\frametitle{Probability Mass Function}
(Formally defining something mentioned previously)
<p> * a probability mass function (pmf) is a \textbf{\emph{function}}
that gives the probability that a discrete random variable is exactly equal to some
value.
\[P(X=k)\]
* The probability mass function is often the primary means of defining a discrete
probability distribution
* It is conventional to present the probability mass function in the form of a table.
* The p.m.f of a value $k$ is often denoted $f(k)$.
\end{itemize}
}
%--------------------------------------------------------------------------------------%

\frametitle{ Binomial Example 1 }
(Revision from Last Class)\\
Suppose a die is tossed 5 times. What is the probability of getting exactly 2 fours?

\textbf{Solution:} This is a binomial experiment in which the number of trials is equal to 5, the number of successes is equal to 2, and the probability of success on a single trial is 1/6 or about 0.167. 
\\
\bigskip
Therefore, the binomial probability is:

\[P(X=2) = ^5C_2 \times (1/6)^2 \times (5/6)^3 = 0.161\]
}

%--------------------------------------------------------------------------------------%

\frametitle{ Binomial Example 2 }
Suppose there is a container that contains 6 items.  The probability that any one of these items is defective is 0.3. Suppose all six items are inspected. 
<p>
* What is the probability of 3 defective components?
* What is the probability of 4 defective components?
\end{itemize}

\[ P(3\text{ defects}) = f(3) = P(X = 3) = {6\choose 3}0.3^3 (1-0.3)^{6-3} = 0.1852 \]
\[ P(4\text{ defects}) = f(4) = P(X = 4) = {6\choose 4}0.3^4 (1-0.3)^{6-4} = 0.0595 \]
}

%--------------------------------------------------------------------------------------%

\frametitle{Probability Tables}
In the \textbf{Sulis} workspace there are two important tables used for this part of the course.


This class will feature a demonstration on how to read those tables.
<p>
* The Cumulative Binomial Tables (Murdoch Barnes Tables 1)
* The Cumulative Poisson Tables (Murdoch Barnes Tables 2)
\end{itemize}

Please get a copy of each as soon as possible.

}

%---------------------------------------------------------------------------%

\frametitle{Probability Tables}
<p>
* For some value $r$ the tables record the probability of $P(X \geq r)$.
* The Student is required to locate the appropriate column based on the parameter values for the distribution in question.
* A copy of the Murdoch Barnes Tables will be furnished to the student in the End of Year Exam. The Tables are not required for the first mid-term exam.
* Knowledge of the sample space, partitioning of the sample points, and the complement rule are advised.
\end{itemize}
}
%---------------------------------------------------------------------------%



\frametitle{Binomial Distribution : Using Tables}
It is estimated by a particular bank that 25\% of credit card customers pay only the minimum amount due on their monthly credit card bill and do not pay the total amount due. 50 credit card customers are randomly selected.
\begin{enumerate}
* (3 marks)	What is the probability that 9 or more of the selected customers pay only the minimum amount due?
* (3 marks) What is the probability that less than 6 of the selected customers pay only the minimum amount due?
* (3 marks)	What is the probability that more than 5 but less than 10 of the selected customers pay only the minimum amount due?
\end{enumerate}

}


\frametitle{Binomial Distribution : Using Tables}
Demonstration on Blackboard re: how to use tables in class.
\begin{enumerate}
* $P(X \geq 9) = 0.9084$
* $P(X < 6) = 1- P(X \geq 6) =1 - 0.9930 = 0.0070$
* $P(6 \leq X \leq 9) = P(X \geq 6) - P(X \geq 10) = 0.9930 - 0.8363 = 0.1567$
\end{enumerate}

}


%------------------------------------------------------------------%

\frametitle{Binomial Distribution: Expected Value and Variance}


If the random variable X has a binomial distribution with parameters n
and p, we write
\[ X \sim B(n,p) \]

Expectation and Variance
If $X \sim B(n,p)$, then:

<p>
* Expected Value of X : $E(X) = np$
* Variance of X : $Var(X) = np(1-p)$
\end{itemize}

Suppose n=3 and p=0.5 
Then $E(X) = 1.5$ and $V(X) = 0.75$.

Remark: Referring to the expected value and variance may be used to validate
the assumption of a binomial distribution.

}
%---------------------------------------------------------------------------%

\frametitle{The Geometric Distribution}
<p>
* The Geometric distribution is related to the Binomial distribution in that
both are based on independent trials in which the probability of success
is constant and equal to p.
* However, a Geometric random variable is the number of trials until the
first failure, whereas a Binomial random variable is the number of
successes in n trials.
* The Geometric distributions is often used in IT security applications.
\end{itemize}
}
%---------------------------------------------------------------------------%

\frametitle{The Geometric Distribution}

Suppose that a random experiment has two possible outcomes, success
with probability p and failure with probability 1-p .


The experiment is repeated until a success happens. The number of
trials before the success is a random variable X computed as follows

\[P(X = k) = (1-p)^{(k-1)}\times p \]


(i.e. The probability that first success is on the k-th trial)
}


%---------------------------------------------------------------------------%

\frametitle{The Geometric Distribution: Notation}

If X has a geometric distribution with parameter p, we write
\[X \sim Geo(p) \]
Expectation and Variance
If $X \sim Geo(p)$, then:

<p>
* Expected Value of X : E(X) = 1/p
* Variance of X : Var(X) = $(1-p)/p^2$.
\end{itemize}
}




%---------------------------------------------------------------------%

\frametitle{The Cumulative Distribution Function}
<p>
* The Cumulative Distribution Function, denoted $F(x)$, is a common way that the probabilities
of a random variable (both discrete and continuous) can be summarized.
* The Cumulative Distribution Function, which also can be
described by a formula or summarized in a table, is defined as:
\[F(x) = P(X \leq x) \]
* The notation for a cumulative distribution function, F(x), entails using a capital
"F".  (The notation for a probability mass or density function, f(x), i.e. using a lowercase "f". The notation is not interchangeable.
\end{itemize}


%---------------------------------------------------------------------%

\frametitle{Useful Results}
(Demonstration on the blackboard re: partitioning of the sample space, using examples on next slide)
<p>
* $P(X \leq 1) = P(X=0) + P(X=1)$
* $P(X \leq r) = P(X=0)+ P(X=1) + \ldots P(X= r)$
* $P(X \leq 0) = P(X=0)$
* $P(X = r) = P(X \geq r ) - P(X \geq r + 1)$
* \textbf{Complement Rule}: $P(X \leq r-1) = P(X < r) = 1 - P(X \geq r)$
* \textbf{Interval Rule}:$ P(a \leq X \leq  b)= P(X \geq a) - P(X \geq b + 1).$
\end{itemize}
For the binomial distribution, if the probability of success is greater than 0.5, instead of
considering the number of successes, to use the table we consider
the number of failures.

%---------------------------------------------------------------------%


%---------------------------------------------------------------------%

\frametitle{Binomial Example 1}
Suppose a signal of 100 bits is transmitted and the probability of
sending a bit correctly is 0.9. What is the probability of
\begin{enumerate}
* at least 10 errors
* exactly 7 errors
* Between 5 and 15 errors (inclusively).
\end{enumerate}

%---------------------------------------------------------------------%

\frametitle{Binomial Example 1}
<p>
* Since the probability of success is 0.9. We consider the distribution
of the number of failures (errors).
* We reverse the definition of `success' and `failure'. Success is now defined as an error.
* The probability that a bit is sent incorrectly is 0.1.
* Let X be the total number of errors. $X \sim B(100, 0.1)$.
* Answer : $P(X \geq 10) = 0.5487$.
* $P(X = 7)=P(X \geq 7) - P(X \geq 8) =0.8828 - 0.7939 = 0.0889$.
* $P(5 \leq X  \leq 15) = P(X \geq 5) - P(X \geq 16) =0.9763 - 0.0399 = 0.9364$
\end{itemize}



\end{document}

%---------------------------------------------------------------------------%

\frametitle{Binomial Probability Distribution: Example 2}
The vice-president of a computer firm has reviewed the records of the firm�s personnel and has found that 70\% of the employees read a well known industry magazine ``The IT Journal". \\ \bigskip
If the vice-president was to choose 10 employees at random, what is the probability that the number of these employees who do not read the ``IT Journal" is the following?
\normalsize
<p>
* [1] At least five.
* [2] Between four and eight, inclusive.
* [3] No more than seven.
% * [4] What are the mean and variance of this distribution?
\end{itemize}
}

%---------------------------------------------------------------------------%
[fragile]
\frametitle{Binomial Probability Distribution: Example 2}


<p>
\item
Firstly, identify the probability distribution to be used?
    <p>
    \item
    Answer: the binomial distribution
    \end{itemize}
* We are given the number of trials ( `` choose 10 employees")

* We are given a definition of a ``success", which is finding an employee that did NOT reads the journal.

* We are given the probability of such a success : 30\%  or 0.30

* So our binomial parameters are n= 10 and p = 0.30

* Let's use the following \texttt{R} code to solve.

\end{itemize}

%---------------------------------------------------------------------------%
[fragile]
\frametitle{Binomial Probability Distribution: Example 2}

\begin{verbatim}
> 0:10
 [1]  0  1  2  3  4  5  6  7  8  9 10
>
> dbinom(0:10,size=10,prob=0.30)
 [1] 0.0282475249 0.1210608210 0.2334744405 0.2668279320
 [5] 0.2001209490 0.1029193452 0.0367569090 0.0090016920
 [9] 0.0014467005 0.0001377810 0.0000059049
>
> pbinom(0:10,size=10,prob=0.30)
 [1] 0.02824752 0.14930835 0.38278279 0.64961072 0.84973167
 [6] 0.95265101 0.98940792 0.99840961 0.99985631 0.99999410
[11] 1.00000000
\end{verbatim}

%---------------------------------------------------------------------------%
[fragile]
\frametitle{Binomial Probability Distribution: Example 2}
Question 1: Probability of at least five $P(X \geq 5)$.

\bigskip We have already determined the probability of the complement event $P(X \leq 4)$, which is $84.97\%$. Therefore the answer is $P(X \geq 5)$ = $15.03\%$.

\begin{verbatim}
> pbinom(0:10,size=10,prob=0.30)
 [1] 0.02824752 0.14930835 0.38278279 0.64961072 0.84973167
 [6] 0.95265101 0.98940792 0.99840961 0.99985631 0.99999410
[11] 1.00000000
\end{verbatim}


%---------------------------------------------------------------------------%
[fragile]
\frametitle{Binomial Probability Distribution: Example 2}
Question 2: Probability of between 4 and 8 inclusive $P(4 \leq X \leq 8)$.

\bigskip Lets look at the sample space again, with the relevant sample points in bold: $S = \{0,1,2,3,\textbf{4,5,6,7,8},9,10,\}$.
\begin{verbatim}
> pbinom(0:10,size=10,prob=0.30)
 [1] 0.02824752 0.14930835 0.38278279 0.64961072 0.84973167
 [6] 0.95265101 0.98940792 0.99840961 0.99985631 0.99999410
[11] 1.00000000
\end{verbatim}
<p>
* $P(X \leq 8)$ is $99.98\%$, but this includes the probability of $X = \{0,1,2,3\}$
* We can simply subtract $P(X \leq 3)$ from $P(X \leq 8)$ to get the desired value.
* $(P 4 \leq X \geq 8)$ = $99.98\%$ - $64.96\%$ = $35.02\%$
\end{itemize}



%---------------------------------------------------------------------------%
[fragile]
\frametitle{Binomial Probability Distribution: Example 2}
Question 3: Probability of no more than 7 $P(X \leq 7)$.
\begin{verbatim}
> pbinom(0:10,size=10,prob=0.30)
 [1] 0.02824752 0.14930835 0.38278279 0.64961072 0.84973167
 [6] 0.95265101 0.98940792 0.99840961 0.99985631 0.99999410
[11] 1.00000000
\end{verbatim}
\[ P(X \leq 7) = 99.84\% \]

%---------------------------------------------------------------------------%