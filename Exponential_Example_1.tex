
%--------------------------------------------------------------------------------------%
\begin{frame}
\frametitle{The Exponential Distribution}
\begin{itemize}
\item The exponential distribution is a continuous probability distribution commonly used to model durations or ``lifetimes".
\item A lifetime could mean
\begin{itemize}
\large
\item the lifespan of a component
\item the time it takes to complete a task
\item the amount of time between two successive occurrences, such as withdrawals from a bank machine.
\end{itemize}
\item The average lifetime is denoted $E(X) = \mu$.
\item The variance of lifetimes is computed as $V(X) = \mu^2$
\end{itemize}
\end{frame}

%--------------------------------------------------------------------------------------%
\frame{
\frametitle{Important Formulae}
\Large
The probability that a lifetime $X$ will be less than a period of $k$ time units is given by
\[
P( X \leq k) = 1- e^{{-k \over \mu}}.
\]
Similarly, the probability that a lifetime $X$ will be greater than a period of $k$ time units is given by
\[
P( X \geq k) = e^{{-k \over \mu}}.
\]
}





%----------------------------------------------------------------------------%
\frame{
\frametitle{The Exponential Distribution}
A continuous random variable having p.d.f. f(x), where:
$f(x) = \lambda x e ^{-\lambda x} $
is said to have an exponential distribution, with parameter $\lambda$.
The cumulative distribution is given by:
$F(x) = 1 - e^{\lambda x}$

Expectation and Variance
$E(X) = 1 / \lambda$\\
$V(X) = 1 / \lambda^2$\\
}

%----------------------------------------------------------------------------%
\frame{
\frametitle{Example}
Suppose that the service time for a customer at a fast-food outlet
has an exponential distribution with mean 3 minutes. What is the probability that a
customer waits more than 4 minutes?

\[ P(X  \leq 4) = 1 -  e^{-4/3} \]

\[ P(X  \leq 4) = e^{-4/3} = 0.2636 \]
}


%---------------------------------------------------------------------------------%
\begin{frame}
\frametitle{Exponential Distribution Lifetimes}
The average lifespan of a laptop is 5 years. You may assume that
the lifespan of computers follows an exponential probability
distribution. \begin{itemize}\item (3 marks) What is the
probability that the lifespan of the laptop will be at least 6
years? \item
What is the probability that the lifespan of the laptop will not
exceed 4 years? \item What is the probability of the
lifespan being between 5 years and 6 years?
\end{itemize}
Suppose the lifetime of a PC is exponentially distributed with
mean $\mu =5$
We should be told the average lifetime $\mu$.
\[
P( X \geq x_o) = e^{{-x_o \over \mu}}
\]
\end{frame}

\begin{frame}
	\frametitle{Exponential Distribution Lifetimes}

\begin{itemize}
\item What is the probability that the lifespan of the laptop will be at least
6 years?
\item What is the probability that the lifespan of the laptop will not exceed
4 years?
\item What is the probability of the lifespan being between 5 years and 6
years?
\end{itemize}

\end{frame}

\end{document}
